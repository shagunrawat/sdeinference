\documentclass[12pt]{article}
\usepackage[margin=1.25in]{geometry}
\usepackage{amsmath,amssymb,latexsym}

\begin{document}
\begin{center}
Spectral DTQ Method, May 2017
\end{center}
We start with the SDE in $\mathbb{R}^N$:
$$
dX_t = f(X_t) dt + g dW_t
$$
Here $g$ is an $N \times N$ invertible matrix.
Let
$$
\phi(y) = y + f(y) h.
$$
Assuming that $f$ is Lipschitz we see that $\phi$ is invertible for sufficiently small $h$; this follows from
$$
D \phi(y) = I + Df(y) h.
$$
Discretizing the SDE in time, we obtain
$$
X_{n+1} = X_n + f(X_n) h + g h^{1/2} Z_{n+1}.
$$
Here $Z_{n+1}$ is a sequence of independent multivariate Gaussians, each
with mean vector $0$ and covariance matrix equal to the identity matrix $I$.
Hence $X_{n+1}$ given $X_n = y$ has multivariate Gaussian density with mean
vector $\phi(y)$ and covariance matrix $h g g^T$.
Now moving from sample paths to densities, we have
$$
\widetilde{p}(x,t_{n+1}) = \int_{y \in \mathbb{R}^N} G(x - \phi(y);h g g^T) \widetilde{p}(y,t_n) \, dy.
$$
Let
$$
G(w;h g g^T) = \frac{1}{\sqrt{ (2 \pi h)^N |g|^2 }} \exp \left( -\frac{1}{2h} w^T (g g^T)^{-1} w \right).
$$
Now we let $z = \phi(y)$ so that $dz = \det D\phi(y) \, dy$.  Then
$$
\widetilde{p}(x,t_{n+1}) = \int_{z \in \mathbb{R}^N} G(x - z; h g g^T) \underbrace{\widetilde{p}(\phi^{-1}(z),t_n) \, \det [D\phi(\phi^{-1}(z))]^{-1}}_{\psi_n(z)} \, dz.
$$
Let $\{q_m\}$ be a set of collocation points, and let $K$ be a kernel function.  We expand:
$$
\widetilde{p}(y, t_n) = \sum_m \alpha_m^n K(\phi(y) - q_m) \det [D\phi(y)] 
$$
Then
$$
\widetilde{p}(x, t_{n+1}) = \sum_m \alpha_m^n \int_{z \in \mathbb{R}^N} G(x - z; h g g^T) K(z - q_m) \, dz.
$$
The point of all this manipulation is to obtain a convolution on the right-hand side.  Now take the Fourier transform of both sides to obtain
$$
\widehat{\widetilde{p}}(k, t_{n+1}) = \sum_m \alpha_m^n \widehat{G}(k,hgg^T) \widehat{K}(k) e^{-2 \pi i q_m k}.
$$
The point is that for a suitable choice of kernel $K$, we should be able to compute $\widehat{K}$ by hand.  We can of course compute $\widehat{G}$ by hand.  Hence the entire right-hand side can be determined without any numerical approximation.

Of course, we then use the inverse Fourier transform to compute
$$
\widetilde{p}(x, t_{n+1}) = \int_{k \in \mathbb{R}^N} e^{2 \pi i k x} \widehat{\widetilde{p}}(k, t_{n+1}) \, dk.
$$
Next, we use the collocation relationship to solve for $\alpha_m^{n+1}$:
$$
\widetilde{p}(x, t_{n+1}) = \sum_m \alpha_m^{n+1} K(\phi(x) - q_m) \det [D\phi(x)] 
$$
Namely, by requiring this equation to hold at $m$ distinct points $x$, we obtain a system of $m$ equations in $m$ unknowns.  We can write this as a matrix-vector system and then solve for $\alpha^{n+1}$.

\end{document}



